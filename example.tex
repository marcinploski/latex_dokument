\documentclass[11pt,twoside,a4paper]{article}
\usepackage[left=20mm, top=20mm, bottom=20mm, right=20mm]{geometry}
\usepackage{latexsym}
\usepackage[polish]{babel}
\usepackage[T1]{fontenc}

% dokument w języku polskim
\usepackage[polish]{babel} 
\usepackage[MeX]{polski} 
\usepackage[utf8]{inputenc}
\usepackage[T1]{fontenc}

% wykorzystanie grafiki
\usepackage{graphicx}
\graphicspath{ {img/} }

% wzory matematyczne
\usepackage{amsmath}

% bibliografia
\usepackage{biblatex}

\author{Marcin Płoski}

\begin{document}
\title{Przykładowy dokument w Latex}
\maketitle

\section{Formatowanie tekstu}
	Inwokacja to apostrofa, najczęściej rozbudowana, otwierająca utwór literacki – w tym wypadku epopeję jaką jest Pan Tadeusz Adama Mickiewicza:	
	\textsl{Litwo! Ojczyzno moja! Ty jesteś jak zdrowie,
Ile cię trzeba cenić, ten tylko się dowie,
Kto cię stracił. Dziś piękność twą w całej ozdobie
Widzę i opisuję, bo tęsknię po tobie
Panno święta, co Jasnej bronisz Częstochowy
I w Ostrej świecisz Bramie! Ty, co gród zamkowy
Nowogródzki ochraniasz z jego wiernym ludem!
Jak mnie dziecko do zdrowia powróciłaś cudem,
(Gdy od płaczącej matki pod Twoją opiekę
Ofiarowany, martwą podniosłem powiekę
I zaraz mogłem pieszo do Twych świątyń progu
Iść za wrócone życie podziękować Bogu),
Tak nas powrócisz cudem na Ojczyzny łono.
Tymczasem przenoś moją duszę utęsknioną
Do tych pagórków leśnych, do tych łąk zielonych,
Szeroko nad błękitnym Niemnem rozciągnionych;
Do tych pól malowanych zbożem rozmaitem,\newline
Wyzłacanych pszenicą, posrebrzanych żytem;
Gdzie bursztynowy świerzop, gryka jak śnieg biała,
Gdzie panieńskim rumieńcem dzięcielina pała,
A wszystko przepasane jakby wstęgą, miedzą
Zieloną, na niej z rzadka ciche grusze siedzą.}

\section{Równanie matematyczne}
\begin{center}
Przykład równania w jednej linii - twierdzenie Pitagorasa:\newline
$a^2 + b^2 = c^2$
\end{center}

\begin{center}
Bardziej rozbudowane równanie:\newline
\begin{equation*}
\lim_{n \to \infty}
\sum_{k=1}^n \frac{1}{k^2}
= \frac{\pi^2}{6}
\end{equation*}
\end{center}

\begin{center}
Wykorzystanie array w równaniu matematycznym:
\begin{equation*}
\label{eq:funkcjaf}
f(x) = \left\lbrace
\begin{array}{rcl}
-x^2 & \text{dla} & x < 0,\\
\sqrt{x} + \sin x & \text{dla} & x > 0.
\end{array}
\right.
\end{equation*}
\end{center}

\begin{center}
Równanie w dwóch liniach:
\begin{align*}
p(x) = 3x^6 + 14x^5y + 590x^4y^2 + 19x^3y^3\\ 
- 12x^2y^4 - 12xy^5 + 2y^6 - a^3b^3
\end{align*}
\end{center}
\newpage

\section{Wykorzystanie obrazka}
\begin{figure}[ht]
\begin{center}
\includegraphics[width=7cm]{logo}
\caption{Logo Latex}{Przykładowe użycie obrazka}
\label{rys_model}
\end{center}
\end{figure}

\section{Bibliografia}
\begin{thebibliography}{99}
\bibitem{pa} Tobias Oetiker:
Nie za krótkie wprowadzenie do systemu LaTeX
\end{thebibliography}

\end{document}